\section{QUA-Compliance}
\label{ch:quacompliance}

To make a service compliant for use in the qua-kit, a service is subject to additional constraints that are described here.
A qua-view-compliant service must support one or more of following working modes:
\begin{itemize}
\item
    \texttt{scenario} -- a service gets a scenario as input and returns a single result for a whole scenario;
\item
    \texttt{objects} -- a service gets a scenario and ids of individual objects and returns a result per every object;
\item
    \texttt{points} -- a service gets a scenario and a grid of points, and returns a result per each point on the grid;
\item
    \texttt{new} -- a service updates or creates a new geometry.
\end{itemize}

\subsection{QUA-View-Compliance}

Defining a service being
\texttt{qua-view-compliant} means promising to follow a strict predefined format of service input, output, and parameters.
To declare a service \texttt{qua-view-compliant}, one needs to add \texttt{qua-view-compliant:true} flag in the root
of \texttt{RemoteRegister} JSON message sent to luci/helen at the start of a service execution.
To test if a service follows strictly the specification is a responsibility of a service developer.

\subsubsection{Optional parameters}

A \texttt{qua-view-compliant} service may have arbitrary number of optional parameters.
All optional parameters must be defined in an \texttt{inputs} object of \texttt{RemoteRegister} JSON message
using \texttt{paramName:paramType} syntax.
Valid parameter types are:
\begin{enumerate}
\item \texttt{boolean} -- either \texttt{true} or \texttt{false};
\item \texttt{number} -- any number type;
\item \texttt{string} -- a fixed string, must have a corresponding field in \texttt{constraints} object.
\end{enumerate}

\subsubsection{Constraints}

Parameters may have constraints defined on them.
These constraints are used by qua-view control panel to configure user input fields.
To define a constraint, one needs to use \texttt{constraints} object in the root of \texttt{RemoteRegister} JSON message
(see listing~\ref{lst:quaconstraints}).
\begin{lstlisting}[caption={qua-compliance -- parameter constraints}, label={lst:quaconstraints}]
{
  run : "RemoteRegister",
  ...
  qua-view-compliant : true,
  inputs : {
    ...
    mode : "string",
    constrainedInput1 : "number",
    constrainedInput2 : "string"
  },
  outputs : {
    ...
  },
  constraints : {
      mode: ["points", "scenario"],
      constrainedInput1 : {
          min : 42,
          integer : true
      },
      constrainedInput2 : ["hello", "world", "3rd option"]
  },
  ...
}
\end{lstlisting}

\paragraph{Number}
To describe constraints on a number type, one needs to create a JSON object with following (all optional) records:
\begin{itemize}
\item \texttt{min:number} -- minimum allowed value;
\item \texttt{max:number} -- maximum allowed value;
\item \texttt{def:number} -- default value visible in a viewer;
   \texttt{\begin{itemize}
   \item (min = undef, max = undef, def = undef): def = 0;
   \item (min = exist, max = undef, def = undef): def = min;
   \item (min = undef, max = exist, def = undef): def = max;
   \item (min = exist, max = exist, def = undef): def = (max - min) / 2;
   \end{itemize}}
\item \texttt{integer:boolean} -- whether to treat a number as an integer; default \texttt{integer:false}.
\end{itemize}

\paragraph{String}
To describe contraints on a string, one needs to create a JSON array of strings -- an enumeration of possible value.
Without a constraint object, a parameter accepts any string value;
with a constraint object, a parameter is displayed as a drop-down list of possible options
(see listing~\ref{lst:quaconstraints} constraints on a \texttt{constrainedInput2} input key).
The first option in a list is the default one.


\paragraph{Boolean}
One possible constraint for a boolean is a default value in qua-view.
Use \texttt{\{def:true\}} or \texttt{\{def:false\}} constructs.


\subsection{Execution modes}
The execution modes define how a service is visualized in qua-view.
Each mode dictates obligatory input and output parameters.
A service may support more than one execution modes.
At runtime, an execution mode is defined by \texttt{inputs} record \texttt{mode:string}.
A qua-view-compliant service must define \texttt{inputs/mode} field in \texttt{RemoteRegister} JSON message
and a corresponding constraint (the same way as optional parameters).

\subsubsection{points}
In this mode, qua-view generate a grid of points in 3D and sends them to a service together with scenario reference.
A service sends back floating point values for each point.
Then, qua-view visualizes the values as a coloured heat map.
The grid of points is sent in a binary attachment in form \texttt{XYZXYZ...XYZ}, 4 byte floating point value per coordinate
in Little Endian format (\texttt{float32}) -- $3 \cdot 4 n$ bytes in total, where $n$ is the number of points.
A service determines a number of points by the size of incoming data.
\begin{lstlisting}[caption={A qua-compliant service run request for mode \texttt{points}}, label={lst:quacompliantinput:points}]
{
  run : "RandomQUA",
  callID : 283,
  ScID : 123,
  mode : "points",
  points : {
    format : "float32 array",
    attachment {
      length : 3072,
      position : 1,
      checksum : "2929bead3ee3cf55113ec9aade2b6add"
    }
  }
}
\end{lstlisting}

\begin{lstlisting}[caption={A qua-compliant service output for mode \texttt{points}}, label={lst:quacompliantresult:points}]
{
  callID : 283,
  result : {
    units : "metre",
    values : {
      format : "float32 array",
      attachment : {
        length: 1024,
        checksum: "50752c7cb358a5fffd913d9aa3605433",
        position: 1
      }
    }
  }
}
\end{lstlisting}
A result of a service execution must contain following fields:
\begin{enumerate}
\item \texttt{units} -- any string, e.g. "metre", "kg", "mm/s", etc.;
\item \texttt{values} -- attachment object containing a reference to a blob of \texttt{float32} data;
 its size must be exactly $4 n$, where $n$ is the number of points received.
\end{enumerate}

\subsubsection{objects}

This mode is similar to \texttt{points} mode except instead of point coordinates a client (qua-view) sends ids of geometric objects
within a scenario.
Qua-view visualizes it by colouring objects according to the values received for each object.
A user also can click on a building and see exact numeric value received from a service.

A service in this mode has two obligatory inputs:
\begin{enumerate}
\item \texttt{ScID} -- id of a current scenario;
\item \texttt{objIds} -- an attachment reference to a binary blob of size $8n$, where $n$ is the number of objects to process by a service;
                         the blob consists of unsigned 8-byte integers in Little Endian format (\texttt{ulong64}) -- \texttt{geomID} of each object.
\end{enumerate}
\begin{lstlisting}[caption={A qua-compliant service run request for mode \texttt{objects}}, label={lst:quacompliantinput:objects}]
{
  run : "RandomQUA",
  callID : 172,
  ScID : 123,
  mode : "objects",
  objIds : {
    format : "ulong64 array",
    attachment {
      length : 2048,
      position : 1,
      checksum : "50752c7cb358a5fffd913d9aa3605433"
    }
  }
}
\end{lstlisting}

A result of a service execution must contain following fields:
\begin{enumerate}
\item \texttt{units} -- any string, e.g. "metre", "kg", "mm/s", etc.;
\item \texttt{values} -- attachment object containing a reference to a blob of \texttt{float32} data;
 its size must be exactly $2 n$, where $n$ is the number of objects' \texttt{geomID} received.
\end{enumerate}
\begin{lstlisting}[caption={A qua-compliant service output for mode \texttt{objects}}, label={lst:quacompliantresult:objects}]
{
  callID : 172,
  result : {
    units : "metre",
    values : {
      format : "float32 array",
      attachment : {
        length: 1024,
        checksum: "2929bead3ee3cf55113ec9aade2b6add",
        position: 1
      }
    }
  }
}
\end{lstlisting}

\subsubsection{scenario}
A scenario execution mode returns only one value for a whole scenario.
The value can be an arbitrary string or a \texttt{.png} image (in attachment).
In either way, the result is displayed on a side panel in qua-view.

The only obligatory field in this mode is a scenario id \texttt{ScID}.
\begin{lstlisting}[caption={A qua-compliant service run request for mode \texttt{scenario}}, label={lst:quacompliantinput:scenario}]
{
  run : "RandomQUA",
  callID : 213,
  ScID : 125,
  mode : "scenario"
}
\end{lstlisting}
One possible result of a scenario-mode-service is a text string.
\begin{lstlisting}[caption={A qua-compliant service output for mode \texttt{scenario} (string output)}, label={lst:quacompliantresult:scenario}]
{
  callID : 213,
  result : {
    answer : "This scenario looks good! The goodness level is 9.125."
  }
}
\end{lstlisting}

Another possible result type is a \texttt{.png} image attached to a message.
Recommended size of an image is 512x512px.
An image may contain a graph or an arbitrary figure.
\begin{lstlisting}[caption={A qua-compliant service output for mode \texttt{scenario} (image output)}, label={lst:quacompliantresult:scenario}]
{
  callID : 213,
  result : {
    image : {
      format : "image/png",
      attachment : {
        length: 12623,
        checksum: "8329beex3ee3cf55113ec9aade2b6a02a",
        position: 1
      }
    }
  }
}
\end{lstlisting}


\subsubsection{new}

In this mode, a service updates or creates a new geometry.
It can be used by various geometry-generating services.
A \texttt{ScID} key is optional. This mode does not have obligatory parameters.
A service should write its output directly into scenario via luci/helen.
The output of a service must have following fields:
\begin{lstlisting}
result: {
  ScID : long, // the id of the newly created scenario
  timestamp_accessed : long, // optional - the timestamp at which the scenario was accessed if the ScID-field was set in the input
  timestamp_modified : long // the timestamp at which the new scenario was stored
}
\end{lstlisting}

\begin{lstlisting}[caption={A qua-compliant service run request for mode \texttt{new}}, label={lst:quacompliantinput:new}]
{
  run : "RandomQUA",
  callID : 244,
  mode : "new" // Note how the call does not contain a ScID
}
\end{lstlisting}

\begin{lstlisting}[caption={A qua-compliant service output for mode \texttt{new}}, label={lst:quacompliantresult:new}]
{
  callID : 244,
  result : {
    ScID : 1337,
    timestamp_accessed : 1472137755,
    timestamp_modified : 1472137758
  }
}
\end{lstlisting}

\subsection{Registering a service}

Listing \ref{lst:quacompliance} shows a registration message of a qua-compliant service named \emph{RandomQUA}. In the following, we will briefly reiterate what each field indicates.

\begin{itemize}
  \item Lines 1-5: A service of name \emph{RandomQUA} is registered. The fields are not subject to any change by qua-compliance. See Section \ref{ch:serviceapi}, for how to register a generic service.
  \item Line 6: The field \texttt{qua-view-compliant} indicates that the service is qua-view-compliant.
  \item Lines 7-14 specify the service's inputs
    \begin{itemize}
      \item The field \texttt{ScID} is optional, it is used to specify the scenario on which the service is asked to operate. Since not every execution mode requires such a scenario, the field is optional.
      \item The execution mode \texttt{mode} is obligatory and, as can be noted from the constraints in lines 23-33, takes the four execution mode names \emph{points, objects, scenario} and \emph{new}.
      \item The field \texttt{points} is an optional binary attachment that is required when the service is executed in the \emph{points}-execution-mode
      \item The field \texttt{objIds} is an optional binary attachment specifying the object ids when the service is executed in the \emph{objects}-execution-mode
      \item The other fields are service-specific fields, in this case two required numbers. Generally, a service can add an arbitrary amount of additional inputs for the service.
    \end{itemize}
  \item Lines 15-22 specify the service's outputs
    \begin{itemize}
      \item The field \texttt{unit} gives the unit of the service result
      \item The mutually exclusive fields \texttt{value} or \texttt{values} contain the service's result value(s).
        If the service operated in the \emph{points} or \emph{objects} execution modes.
        The field \emph{values} will contain a binary attachment in which the result values are encoded as floating point numbers.

        On the other hand, if the service operated in the \emph{scenario}-mode, the field \texttt{value} will contain the result for the entire scenario.
      \item The fields \texttt{scenario\_id}, \texttt{timestamp\_accessed} and \texttt{timestamp\_modified} describe the outputs if the service has been run in the \emph{new}-mode. They describe the scenario id, its version number (timestamp) when it was accessed and its version number after it has been modified, respectively.
    \end{itemize}
  \item Lines 23-33 specify the service's input constraints
    \begin{itemize}
      \item Lines 24-26: The input argument ScID must be an integer
      \item Line 27: The mode must be either \emph{points}, \emph{objects}, \emph{scenario} or \emph{new}
      \item Line 28-32: The number field \texttt{some\_other\_input} must be within the domain $[42,100] \cap \mathbb{Z}$.
    \end{itemize}
  \item Lines 34-52 give an example of a valid call of the service. In this case, the service is asked to operate on scenario no. $123$. The service is executed in mode \emph{points} length of the \emph{points}-attachment is 36 bytes which corresponds to 3 three-dimensional points. Finally, the two additional input arguments \texttt{some\_other\_input} and \texttt{some\_4th\_input} are set to with parameters 42 and 13.37, respectively.
\end{itemize}

\newpage

\begin{lstlisting}[caption={Registering a QUA-compliant service}, label={lst:quacompliance}]
{
  run : "RemoteRegister",
  description : "I am a random service",
  serviceName : "RandomQUA",
  callID : 1,
  qua-view-compliant : true,
  inputs : {
    OPT ScID : number,
    mode : string,
    OPT points : attachment,
    OPT objIds : attachment,
    some other input : number,
    some 4th input : number
  },
  outputs : {
    unit : string,
    XOR value : number,
    XOR values : attachment,
    OPT scenario_id : long,
    OPT timestamp_accessed : int,
    OPT timestamp_modified : int
  }
  constraint : {
      ScID : {
        integer : true
      }
      mode : ["points", "objects", "scenario", "new"],
      some other input : {
          min : 42,
          max : 100,
          integer : true
      }
  },
  exampleCall : {
    {
      run : "RandomQUA",
      ScID : 123,
      mode : "points",
      points : {
        format : "float32 array",
        attachment {
          length : 36,
          position : 1,
          checksum : "2929bead3ee3cf55113ec9aade2b6add"
        }
      }
      some other input : 42,
      some 4th input : 13.37
    }
  }
}
\end{lstlisting}


\clearpage
